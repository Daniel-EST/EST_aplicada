\documentclass[11pt,a4paper]{article}
\usepackage[utf8]{inputenc}
\usepackage[portuguese]{babel}
\usepackage[T1]{fontenc}
\usepackage[left=1.7cm,right=1.5cm,top=2.5cm,bottom=2cm]{geometry}
\usepackage{xcolor}
\definecolor{orange}{RGB}{255,127,0}
%\usepackage[dvipsnames]{xcolor}
\usepackage{titlesec}
\usepackage[utf8]{inputenc}
\usepackage[portuguese]{babel}
\usepackage[T1]{fontenc}
\usepackage{amsmath}
\usepackage{amsthm}
\usepackage{amsfonts}
\usepackage{amssymb}
\usepackage{graphicx}
\usepackage{stackengine}
\usepackage{accents}
\usepackage{xcolor}
\usepackage{bbm}
\usepackage{enumitem}
\usepackage{mathtools}
\usepackage{ mathrsfs }
\newtheorem{teo}{\underline{Teorema}}
\newtheorem*{defi}{\underline{Definição}}
\newtheorem{prop}{\underline{Proposição}}
\newtheorem{prop2}{\underline{Propriedade}}
\newtheorem*{col}{\underline{Corolário}}
\newtheorem{lema}{\underline{Lema}}
\usepackage{bbm}
\newcommand{\e}{\mathbb{E}}
\newcommand{\var}{\mathrm{Var}}
\newcommand{\cov}{\mathrm{Cov}}
\newcommand{\p}{\mathbb{P}}
\newcommand{\dis}{\displaystyle}
\usepackage{subfigure}
\usepackage{hyperref}
\title{Perfil de Consumo de Usuário de Jogos Eletrónicos}
\author{Carolina Martins, Daniel dos Santos, Fernanda Fernandes, Gabriel Mizuno, Isabelly  Almeida}
\date{ }
 
\begin{document}
 
\maketitle
 
\tableofcontents
 
\section{Introdução}

Com rápido e surpreendente crescimento da industria dos jogos eletrónicos, como foi observado em 2016, onde a final do mundial do jogo League of Legends teve mais telespectadores do que os sete (7) jogos da final da NBA (\href{https://bit.ly/2Dvdtw9}{\textcolor{blue}{LoL vs NBA}}) e em uma outra situação, onde lançamento dos principais jogos tiveram lucros maiores do que Blockters feito em Hollywood  (\href{https://bit.ly/2FfWHTk}{\textcolor{blue}{Filmes vs Jogos}}), a analise do perfil de consumo dos usuários torna-se cada vez mais necessária para conquista e fidelização de seus usurários. Essa pesquisa visa fazer essa analise, de maneira não muito profunda, sobre usuários que vivem no Brasil, 
\\ 
\\
 
\section{Metodologia}

Para a obtenção dos dados foi criado um formulário usando o Google Forms  e durante o período de 1 de Novembro de 2018 ate 13 de Novembro de 2018 foi distribuído em grupo de discussão na plataforma do Facebook. Para a construção do formulário foi usado os seguintes critério:

\begin{enumerate}[label=(\roman*)]
\item Ser jogador independente da plataforma;
\item Ser brasileiro ou morar no Brasil.
\end{enumerate}
Depois de coletados os dados foram utilizados os software estatístico Rstudio e Excel, também foi necessário tratar a base de dados, pois durante a obtenção desses dados alguns voluntários colocaram respostas que não condiziam com a realidade do ser humano, sendo assim foram usados os seguintes critério para retirar essa anomalias.
 \begin{enumerate}[label=(\roman*)]
\item Pessoas com menos de 15 com Ensino Superior ou Médio completo; 
\item Pessoas com menos de 13 com  Ensino Superior ou Médio  ou Fundamental completo;
\item Pessoas com menos de 0 anos ou mais de 100 anos1;
\item Pessoas com menos de 19 anos com Pós-Graduação
\end{enumerate}
Depois de realizado essa filtração tirando cerca de XXXXX voluntários.
\\

Depois de tratada foi iniciada as analises estatísticas, nessas analises foram usados testes de normalidade (Teste de Kolmogorov-Smirnov)

\subsection{Escala Likert}

 
\section{Conclusão}

\section{Apêndice}
\subsection{Glossário}

\begin{enumerate}[label=(\roman*)]
\item \textit{Downloadeable Content} (DLC) = Expansão de jogos, como: mapas, armas, missões extras, etc ;
\item \textit{Free to Play} (F2P) = Jogos gratuitos ;
\item \textit{Role Playing Game} (RPG) = Jogos de interpretação de personagens, como: Dungeons and Dragons, The Witcher, etc ;
\item \textit{First Person Shooter} (FPS) = Jogos de tiro em primeira pessoa, como: Counter - Strike, Overwatch, Call of Duty, etc ;
\item \textit{Third Person Shooter} (TPS) = Jogos de tiro em terceira pessoa, como: Fortnite, PLAYERUNKNOW`S BATTLEGROUND, Gears of War, etc ;
\item \textit{Multiplayer Online Battle Arena} (MOBA) = Jogos baseados em arena de combate, geralmente entre times, como: DOTA 2, League of Legends, Heroes of the Storm, Heroes of Newerth, etc ;
\item \textit{Massive Multiplayer Online Role Playing Game} (MMORPG) = Jogos de interpretação de personagem online com vários outros jogadores, como: World of Warcraft, Black Desert, TERA, Ragnarok, etc ;
\item \textit{Real Time Strategy} (RTS) = Jogos de estratégia em tempo real, como: Starcraft, Age of Empires, etc;
\item \textit{Puzzle} = Jogos de quebra - cabeça, como: Portal , etc;
\end{enumerate}

\end{document}
