\documentclass[11pt,a4paper]{article}
\usepackage[utf8]{inputenc}
\usepackage[portuguese]{babel}
\usepackage[T1]{fontenc}
\usepackage[left=1.7cm,right=1.5cm,top=2.5cm,bottom=2cm]{geometry}
\usepackage{xcolor}
\definecolor{orange}{RGB}{255,127,0}
%\usepackage[dvipsnames]{xcolor}
\usepackage{titlesec}
\usepackage[utf8]{inputenc}
\usepackage[portuguese]{babel}
\usepackage[T1]{fontenc}
\usepackage{amsmath}
\usepackage{amsthm}
\usepackage{amsfonts}
\usepackage{amssymb}
\usepackage{graphicx}
\usepackage{stackengine}
\usepackage{accents}
\usepackage{xcolor}
\usepackage{bbm}
\usepackage{enumitem}
\usepackage{mathtools}
\usepackage{ mathrsfs }
\newtheorem{teo}{\underline{Teorema}}
\newtheorem*{defi}{\underline{Definição}}
\newtheorem{prop}{\underline{Proposição}}
\newtheorem{prop2}{\underline{Propriedade}}
\newtheorem*{col}{\underline{Corolário}}
\newtheorem{lema}{\underline{Lema}}
\usepackage{bbm}
\newcommand{\e}{\mathbb{E}}
\newcommand{\var}{\mathrm{Var}}
\newcommand{\cov}{\mathrm{Cov}}
\newcommand{\p}{\mathbb{P}}
\newcommand{\dis}{\displaystyle}
\usepackage{subfigure}
\usepackage{hyperref}
\title{Perfil de Consumo de Usuário de Jogos Eletrónicos}
\author{Carolina Martins, Daniel dos Santos, Fernanda Fernandes, Gabriel Mizuno, Isabelly  Almeida}
\usepackage{tasks}
\usepackage{exsheets}
\SetupExSheets[question]{type=exam}
\usepackage{makecell}
\usepackage{color, colortbl}
\date{ }
 
\begin{document}
 
\maketitle
 
\tableofcontents

\newpage
\section{Introdução}

Com rápido e surpreendente crescimento da industria dos jogos eletrónicos, como foi observado em 2016, onde a final do mundial do jogo League of Legends teve mais telespectadores do que os sete (7) jogos da final da NBA (\href{https://bit.ly/2Dvdtw9}{\textcolor{blue}{LoL vs NBA}}) e em uma outra situação, onde lançamento dos principais jogos tiveram lucros maiores do que blockbusters feitos em Hollywood  (\href{https://bit.ly/2FfWHTk}{\textcolor{blue}{Filmes vs Jogos}}), a análise do perfil de consumo dos usuários torna-se cada vez mais necessária para conquista e fidelização de seus usurários. Essa pesquisa visa fazer essa analise, de maneira não muito profunda, sobre usuários que vivem no Brasil. 
\\ 
\\
 
\section{Metodologia}

Para a obtenção dos dados foi criado um formulário usando o Google Forms  e durante o período de 1 de Novembro de 2018 até 13 de Novembro de 2018 foi distribuído em grupo de discussão na plataforma do Facebook. Para a construção do formulário foi usado os seguintes critério:

\begin{enumerate}[label=(\roman*)]
\item Ser jogador independente da plataforma;
\item Ser brasileiro ou morar no Brasil.
\end{enumerate}

Depois de coletados os dados foram utilizados os software estatístico Rstudio e Excel, também foi necessário tratar a base de dados, pois durante a obtenção desses dados alguns voluntários colocaram respostas que não condiziam com a realidade do ser humano, sendo assim foram usados os seguintes critério para retirar essa anomalias.
 \begin{enumerate}[label=(\roman*)]
\item Pessoas com menos de 15 com Ensino Superior ou Médio completo; 
\item Pessoas com menos de 13 com  Ensino Superior ou Médio  ou Fundamental completo;
\item Pessoas com menos de 0 anos ou mais de 100 anos;
\item Pessoas com menos de 19 anos com Pós-Graduação
\item Pessoas que responderam Nenhum na pergunta "Quais desses produtos relacionados a jogos você costuma comprar?"  e marcaram alguma outra opções na mesma pergunta.
\end{enumerate}
Usando os critérios citados acima tiramos cerca de 186 voluntários, totalizando assim um total de 1792 voluntários , sendo que desse total obtivemos 183 pessoas do gênero Feminino, 1603 do gênero Masculino e 6 declararam ter um gênero diferentes de Masculino  ou Feminino. 
\\

Depois de tratada foi iniciada as analises estatísticas usando um $\alpha=5\%=0.05$, nessas analises foram usados os seguintes testes:
\begin{itemize}[noitemsep,nolistsep]
\item Kolmogorov–Smirnov
\item Chi-Quadrado (Teste de Independência ou Homogeneidade)
\item ANOVA
\item Levene
\item Bonferroni
\end{itemize}

\subsection{Escala Likert}

A Escala Likert é utilizada para mensurar tendências de uma questão ou afirmação. Utilizamos esse método para avaliar a importância de certas características de jogos. Para isso, dividimos as respostas em 3 itens:

\begin{itemize}[noitemsep,nolistsep]
\item Muita importância; 
\item Média importância;
\item Pouca importância;
\end{itemize}
 
\section{Conclusão}

\section{Apêndice}
\subsection{Tabelas}

\begin{table}[h!]
  \begin{center}
    \begin{tabular}{c|c|c|c|c|c|c|c}
    \hline
      \textbf{Variável} & \textbf{Min} &  \textbf{1º Q} &\textbf{2º Q} &  \textbf{Média}  & \textbf{3º Q} & \textbf{Max}  & \textbf{Desvio Padrão} \\
      \hline
      Idade & 13 & 17 & 20 & 20.96 & 23 & 55 & 5.04\\
      \hline
    \end{tabular}
    \caption{Tabela com medidas resumo da Idade}
     \label{table:1}
  \end{center}
\end{table}

\begin{table}[h!]
 \begin{center}
\begin{tabular}{c|c}
\hline
Genero & Frequencia\\
\hline
Feminino & 183\\
\hline
Masculino & 1603\\
\hline
Outros & 6\\
\hline
\end{tabular}
    \caption{Tabela de frequência por genero}
     \label{table:2}
  \end{center}
\end{table}

\subsection{Gráficos}
\subsection{Glossário}

\begin{enumerate}[label=(\roman*)]
\item \textit{Downloadeable Content} (DLC) = Expansão de jogos, como: mapas, armas, missões extras, etc ;
\item \textit{Free to Play} (F2P) = Jogos gratuitos ;
\item \textit{Role Playing Game} (RPG) = Jogos de interpretação de personagens, como: Dungeons and Dragons, The Witcher, etc ;
\item \textit{First Person Shooter} (FPS) = Jogos de tiro em primeira pessoa, como: Counter - Strike, Overwatch, Call of Duty, etc ;
\item \textit{Third Person Shooter} (TPS) = Jogos de tiro em terceira pessoa, como: Fortnite, PLAYERUNKNOW`S BATTLEGROUND, Gears of War, etc ;
\item \textit{Multiplayer Online Battle Arena} (MOBA) = Jogos baseados em arena de combate, geralmente entre times, como: DOTA 2, League of Legends, Heroes of the Storm, Heroes of Newerth, etc ;
\item \textit{Massive Multiplayer Online Role Playing Game} (MMORPG) = Jogos de interpretação de personagem online com vários outros jogadores, como: World of Warcraft, Black Desert, TERA, Ragnarok, etc ;
\item \textit{Real Time Strategy} (RTS) = Jogos de estratégia em tempo real, como: Starcraft, Age of Empires, etc;
\item \textit{Puzzle} = Jogos de quebra - cabeça, como: Portal , etc;
\end{enumerate}

\subsection{Formulário}

\begin{question}
	Qual é seu gênero?
	\begin{tasks}(3)
		\task Masculino
		\task Feminino
		\task Outros
	\end{tasks}
\end{question}

\begin{question}
	Qual é seu estado civil?
	\begin{tasks}(3)
	\task Solteiro(a)
	\task Casado(a) ou União Estável
	\task Divorciado(a)
	\task Viúvo(a)
	\task Outro
	\end{tasks}
\end{question}

\begin{question}
 	Qual é sua escolaridade?
	\begin{tasks}(3)
	\task Ensino Fundamental - Incompleto
	
	\task Ensino Fundamental - Completo
	\task Ensino Médio - Incompleto
	\task Ensino Médio - Completo
	
	\task Ensino Superior - Incompleto
	
	\task Ensino Superior - Completo
	
	\task Pós - Graduação
	\end{tasks}
\end{question}
\begin{question}
 	Quais gêneros de jogos você joga atualmente?
	\begin{tasks}(5)
	\task FPS
	\task RPG
	\task RTS
	\task MMORPG
	\task MOBA
	\task Battle Royale
	\task Simuladores
	\task Esportes
	\task Luta
	\task Ação e Aventura
	\task Puzzle
	\task Outros
	\end{tasks}
\end{question}
\begin{question}
	Quantas horas por semana você costuma jogar?
	\begin{tasks}(3)
	\task menos de 2 horas
	\task de 2 à 4 horas
	\task de 4 à 6 horas
	\task mais de 6 horas
	\end{tasks}
\end{question}
\begin{question}
	Quais plataformas você possui na sua casa?
	\begin{tasks}(3)
	\task PC (Steam, Origin, Uplay, etc)
	\task Xbox 360 / Xbox One / Xbox One X / Xbox One S
	\task PS3 / PS4 / PS4 Pro
	\task Wii / Wii U / Switch
	\task Celular e Portáteis
	\task Outros
	\end{tasks}
\end{question}
\begin{question}
	Você tem o hábito de comprar jogos durante a pré-venda?
	\begin{tasks}(3)
	\task Sim
	\task Não
	\end{tasks}
\end{question}
\begin{question}
	Você acha justo o preço pago pelos jogos no Brasil?
	\begin{tasks}(3)
	\task Sim
	\task Não
	\end{tasks}
\end{question}
\begin{question}
	Em que tipo de mídia você consome? (compra, presente ou free to play).
	\begin{tasks}(3)
	\task Fisica
	\task Digital
	\end{tasks}
\end{question}
\begin{question}
	Quais desses produtos relacionados a jogos você costuma comprar?
	\begin{tasks}(3)
	\task DLC / Season Pass
	\task Skins
	\task Aprimoramentos (XP Boost, Gold Boost, etc...)
	\task Merchandising (Camisas, Pelúcias, Action Figures, etc...)
	\task Nenhum
	\task Outros
	\end{tasks}
\end{question}
\begin{question}
 Já foi em alguma conferência ou evento relacionadas aos jogos?
	\begin{tasks}(3)
	\task Sim
	\task Não
	\end{tasks}
\end{question}

\begin{question}
	Já gastou mais dinheiro do que tinha ou devia com jogos?
	\begin{tasks}(3)
	\task Sim
	\task Não 
	\task Talvez
	\end{tasks}
\end{question}

\begin{question}
	Você já deixou de jogar por causa dos requisitos mínimos?
	\begin{tasks}(3)
	\task Sim
	\task Não
	\end{tasks}
\end{question}

\begin{question}
\begin{center}
  \begin{tabular}{| c | c | c | c | }
    \hline
    & Alta Importância & Média Importância & Baixa Importância\\ 
    \hline
    \hline
    Jogabilidade & & & \\ \hline
	Gráficos  & & &\\
    \hline
    Multiplayer & & & \\
    \hline
    Multiplayer Competitivo & & & \\
    \hline
    \makecell{	Tempo de Jogo (Tempo \\ necessário para terminar o jogo)}  & & &\\
    \hline
    Grau de Divertimento & & & \\
    \hline
  \end{tabular}
\end{center}
\end{question}
\textcolor{red}{\underline{OBS}: Nas questões 4, 6, 9, 10 e 13 o voluntario poderia marcar mais de uma opção}
\end{document}
